\begin{itemize}
	\item explain Idris \texttt{Nat} numbers (i.e. \texttt{Z} and \texttt{S})?
	\item explain \texttt{Dec}?
	\item TEAMPLAY \texttt{blocks}, \texttt{blockenergy}, \texttt{blocktime},
		  etc?
	\item \texttt{Eq} was already implemented (using `native' things),
		  \texttt{NEq} was more complicated (different ways \texttt{Nat}s can be
		  not equal)
	\item explain how only \texttt{LTE} is required for that selection of
		  operators?
	\item \texttt{And} was already implemented with custom types, used this as
		  inspiration for implementation of \texttt{Or}
	\item explain how these `building blocks'/operators go together (similar to
		  a stack-based language)
\end{itemize}

When writing C programs for embedded systems, the programmer might want to capture certain elements like the time taken or the energy consumed. This is done by declaring C variables, and then using these in the EDSL. When compiling the DSL, the variables are translated to \texttt{Var} types, numbers are translated to \texttt{Lit} types, all of which can then be evaluated using the given \texttt{Env}.
