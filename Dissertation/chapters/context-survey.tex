\begin{itemize}
	\item existing EDSLs focus on code generation or language reasoning, not the
	      HW side
	\item why Haskell and Idris are good tools for implementing EDSLs (citing
    	  \cite{10.1007/3-540-45350-4_5,685738,hudak1996building} for Haskell
    	  and \cite{brady2015embedded} for Idris)
	\item mention Hume \cite{10.1007/978-3-540-39815-8_3} and Feldspar
  		  \cite{5558637} as examples of existing EDSLs
	\item something on how Idris proofs work (citing \cite{brady2015embedded})?
	\item mention previous work done using Idris on concurrency
		  \cite{brady2010correct}?
	\item mention previous work done using Idris on systems programming
		  \cite{10.1007/978-3-642-27694-1_18,Brady:2011:ISP:1929529.1929536}?
	\item mention previous work done using Idris on network protocols
		  \cite{5158855}?
	\item the existing Haskell-based DSLs do not have a built-in proof system
		  (?) for guaranteeing that the ``type-promises'' are correct
	\item outside DSL-scope, things like simulation (cycle-accurate, functional,
		  etc.). But these are slow and/or resource-intensive
	\item the existing Idris-based DSLs do not focus on generating
		  proofs/contracts that the programs do what they say (beyond
		  type-checking) and do not concern themselves with timing or energy
\end{itemize}

\section{Haskell-based DSLs}
	One example of an existing EDSL is ``Feldspar'' \cite{5558637}. Feldspar is a language for reasoning about digital signal processing (DSP) algorithms and is built on Haskell. Its aim is to help build sound DSP algorithms and then use the high-level description to generate fast, low-level C code \cite{5558637}. \textbf{[What/How to elaborate?]}
	
	Another example is ``Hume'' \cite{10.1007/978-3-540-39815-8_3}. The idea behind Hume is that the low-level requirements of the embedded system can guide the process of adapting an existing high(er)-level language to the embedded system, whilst maintaining resource requirements \cite{10.1007/978-3-540-39815-8_3}. \textbf{[Correctly understood? What/How to elaborate?]}

\section{Idris-based DSLs}
	Several Idris-based DSLs exist and demonstrate the benefits of using Idris for implementing the DSLs. State-aware EDSLs for systems programming \cite{10.1007/978-3-642-27694-1_18} allows the programmer to reason about network transfer or correct file-access. This lets them construct safe, low-level protocols in a higher-level language, and the type-checker can help them spot problems by construction \cite{10.1007/978-3-642-27694-1_18} rather than by trial-and-error.
	Another example uses Idris for concurrent programming \cite{brady2010correct}. By representing the state of a concurrent program and the locking and unlocking operations as types, the resulting EDSL allows the programmer to reason about the thread-safety of their program when constructing it, rather than when testing it.
	DSLs can also be used to create correct-by-construction network protocols where the embedded types ensure that the protocol operates correctly as long as it type-checks \cite{5158855}.

\section{Non-DSL Approaches}
	Alternatively to DSLs, a simulator could be used to determine how much time and/or energy a program takes. However, these are often resource intensive and slow. \textbf{[TODO: Citations?]}

\section{Limitations of the existing work}
	DSLs for time and energy constraints in embedded systems exist. However, due to being written in Haskell, there is no built-in, simple way of formally proving that they hold.
	
	The utility of Idris-based DSLs is great: The types ensure that the program is correctly written, and the built-in proof system allows for conditions to be formally guaranteed to hold. Existing research using these properties \cite{10.1007/978-3-642-27694-1_18,brady2010correct,5158855} clearly show that these features can be leveraged to create very powerful prototyping and formal verification methods, but is not in the area of time and energy guarantees.
