For certain system-critical or embedded systems with limited resources, it is often desirable to be able to express and reason about extra-functional properties of programs like time taken or energy used. What is even more desirable would be if the extra-functional properties could be proven to hold or not. This project aims to allow programmers to do this through the use of Embedded Domain Specific Languages (EDSLs) and the \Idris programming language and type system.
\\

Embedded systems are becoming increasingly common. From wireless handsets, smart cards, and routers to scientific sensors, health monitors, and satellites, embedded systems are an increasing part of our daily lives, and control more and more essential parts of it. With more devices connected to the internet than people since approximately 2009 and an estimated 50 billion connected devices by 2020 \cite{cisco-whitepaper}, there is an increasing need to provide certain guarantees for these systems. Since embedded systems often have limited resources, e.g. battery or processing power, being able to reason about how much time and energy programs take would be very useful for the programmers of embedded systems. One way to reason about these are EDSLs.
\\

A DSL is a programming language which is designed for a specific purpose. The benefit to DSLs is that a programmer should be able to much quicker develop the program they want, as the DSL is intended specifically for that task/domain, than they would using a general purpose language \cite{685738}. However, since designing a language is difficult and time-consuming building a language \textit{on top of} another language allows us to use constructs from the ``base'' language (e.g. operators, variable declarations) and use them in our DSL, thereby speeding up its development \cite{hudak1996building}.
\\

DSLs can be used for various purposes. From network protocols \cite{5158855} to concurrency or systems programming \cite{brady2010correct,10.1007/978-3-642-27694-1_18}. Using \Idris as the host language for an EDSL means that the resulting DSL are well-typed. This means we can use the \Idris type-checker and built-in proof system to prove that our DSL is correct and that the timing and energy properties hold. The built-in proof system is based on the \textsc{Ivor} proof engine \cite{10.1007/978-3-540-74130-5_9,brady_2013}. Using existing work by the international \textsc{TeamPlay} project \cite{teamplay:d1.1}, which is part of the EU Horizon 2020 funding programme, this project explores the ability to express and prove timing and energy properties using EDSLs (specifically the \textsc{TeamPlay} framework) and the \Idris programming language \cite{brady_2013}.
\\

The primary objectives of this project are to design a formal framework for assumptions about extra-functional properties of programs by extending the \textsc{TeamPlay} framework to have a complete set of operators; write basic C programs which capture that the operators in the framework function as intended; come up with C programs which use the framework beyond just the operators; and to evaluate how good the \textsc{TeamPlay} project's existing framework is for these purposes.

The secondary objectives of this project are to model ``real-world'' C programs using the framework, and in doing so come up with a collection of programs which showcase how the framework can be used.

The ternary objectives for this project are to explore the modelling of more advanced coding concepts like nested \texttt{for} loops, to automate the translation from C code to the framework, and to support open-ended assertions where the value of one or more of the variables are unknown.
