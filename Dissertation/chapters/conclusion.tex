This dissertation has discussed and evaluated the use of dependent types as a way of proving extra-functional properties of programs for embedded systems. The existing work by the \textsc{TeamPlay} project has been extended to feature a complete set of operators, and these have all been evaluated and shown to work successfully. In evaluating the operators, multiple examples of how annotated programs are translated into \Idris models have been given. The C programs given as part of these show that the \textsc{TeamPlay} annotations seamlessly integrate with the code and are intuitively named.

Programs having more complex features like branches or loops were also successfully provided and modelled, demonstrating that the \textsc{TeamPlay} framework scales to real-life programs.

There is still much work to be done. A compiler for the \textsc{TeamPlay}-annotated C code is currently work in progress by the \textsc{TeamPlay} project, and would drastically improve the time it takes to construct the \Idris models as it would completely automate this. Even more complicated programs with nested loops or branches could be provided. Having these would strengthen the demonstration of the potential of the framework. Finally, expanding the framework to be able to handle open-ended assertions (i.e. assertions where one or more variables are unknown) would be useful as it would enable programmers to get a proof of how much (or how little) margin they have in terms of time or energy. However, doing this is complicated and would require functionality along the lines of a constraint solver for the framework.
