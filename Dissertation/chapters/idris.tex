This chapter aims to give the reader enough of an overview of \Idris and dependent types to be able to understand the work done in the later chapters. A complete explanation of \Idris is beyond the scope of this chapter and project. The chapter borrows a lot of examples and explanations from the \Idris book (``Type-Driven Development with Idris'') \cite{brady_2017}.

\section{Types}

    \begin{code}[caption={In \Idris, the type of a function is specified using `\texttt{:}'}]
        anInt : Int
        anInt = 10
        
        aString : String
        aString = "foo"
        
        aBool : Bool
        aBool = False
    \end{code}

    Types classify values. In programming, we often come across types like \texttt{Int}, \texttt{String}, or \texttt{Bool}, which could be values like \texttt{10}, \texttt{"foo"}, or \texttt{False} respectively.

    \newpage

    \begin{code}[caption={Mismatching types}, escapeinside={(*}{*)}]
        aString : String
        aString = 10
        
        
        -- Compiler error --
        
          |
        2 | aString = 10
          |           ~~
        When checking right hand side of aString with expected type
        String
        
        String is not a numeric type
    \end{code}
    Types in \Idris are checked at compile time, meaning that if the types of the functions in a program do not match (for example, passing an \texttt{Int} where a \texttt{String} is required) then the program will not compile and the compiler will give a type-error.
    
    \begin{code}[label={des:no-cast}, caption={Values are not automatically cast}]
anInt : Int
anInt = 10

half : Double -> Double
half x = x / 2


> half anInt
(input):1:1-10:When checking an application of function Main.half:
        Type mismatch between
                Int (Type of anInt)
        and
                Double (Expected type)
    \end{code}
    
    \Idris is strongly typed, so the compiler will not automatically cast values or parameters. This means you cannot pass an \texttt{Int} to a function which requires a \texttt{Double} and have it automatically work, because \texttt{Int} is a different type, a different `category' from \texttt{Double}.
    
\section{Type Variables and Dependent Types}
    A different example of a type is a list of values. In languages like Java or Python, lists are parametrised over a type.
    \begin{code}[caption={The types of different lists in \Idris}]
        [1, 2, 3, 4, 5] : List Integer
        
        ["a", "b", "c"] : List String
        
        [True, False]   : List Bool
    \end{code}
    In \Idris this is still the case. You can have lists of \texttt{String}, lists of \texttt{Int}, or lists of \texttt{Bool} and these are different types. In general, you can have any \texttt{List elem} where \texttt{elem} is a \textit{type variable} representing the type of the elements of the list. \Idris provides an even more specific type of lists: \texttt{Vect}
    \begin{code}[caption={Example \texttt{Vect} types}]
        [1, 2, 3, 4, 5] : Vect 5 Int
        
        ["a", "b", "c"] : Vect 3 String
        
        [True, False]   : Vect 2 Bool
    \end{code}
    A \texttt{Vect}, short for ``vector'', is a list with a specific length. In general, you can have any \texttt{Vect n elem}. Here, \texttt{elem} is the same as in \texttt{List}, and \texttt{n} is the length of the list. Since the value of \texttt{n} \textit{depends} on the number of elements in the list, we refer to types like \texttt{Vect n elem} as \textit{dependent types} because its precise type depends on other values. Another example of a dependent type is \texttt{Matrix m n elem}, i.e. a matrix of \texttt{m} rows and \texttt{n} columns, with elements of type \texttt{elem}.
    \[
    \begin{pmatrix}
    1 & 2 \\
    3 & 4 \\
    5 & 6 \\
    \end{pmatrix}
    \hspace{50pt}
    \begin{pmatrix}
    ``1" & ``2" & ``3" \\
    ``4" & ``5" & ``6" \\
    \end{pmatrix}
    \]
    Here, the first matrix would have type \texttt{Matrix 3 2 Int} and the second matrix would have type \texttt{Matrix 2 3 String}.
    
    \begin{code}[caption={Types are \textit{first-class}}]
        ty : Type
        ty = Bool
        
        StringOrInt : Bool -> Type
        StringOrInt x = case x of
                            True => Int
                            False => String
    \end{code}
    
    In \Idris, types are \textit{first-class}. This means that they can be used without any restrictions: we can assign them to variables (\texttt{ty}), or have functions return them (\texttt{StringOrInt}). Types are used everywhere, in particular when defining functions.
    
\newpage
    

\section{Functions}
    In Listing \ref{des:no-cast} I defined a function \texttt{half}. Functions in \Idris consist of a \textit{type declaration} and a \textit{function definition}.
    \begin{code}[caption={The \texttt{half} function from Listing \ref{des:no-cast}}]
        half : Double -> Double
        half x = x / 2
    \end{code}
    The first line of the \texttt{half} function is its type declaration and the second line is the function definition. These can each be broken down into further parts. A function definition consists of:
    \begin{itemize}
        \item a function name, \texttt{half}
        \item a function type, \texttt{Double -> Double}
    \end{itemize}
    A function definition is a number of equations with:
    \begin{itemize}
        \item a right hand side, \texttt{half x}
        \item a left hand side, \texttt{x / 2}
    \end{itemize}
    \begin{code}[caption={A different definition of \texttt{half}}]
        half' : Double -> Double
        half' 0.0 = 0.0
        half' n = n / 2
    \end{code}
    In this different definition of \texttt{half}, \texttt{half'}, there are multiple functions definitions. A function can have varying definitions depending on its input. This is known as \textit{pattern matching} and is used extensively in \Idris. Here, it is not massively useful, as all it is doing is not bothering with the division, if the input is 0.

    \begin{code}[caption={A function with a function as an argument}]
        twice : (a -> a) -> a -> a
        twice f x = f (f x)
    \end{code}
    
    Functions in \Idris are first-class, so we can use them as arguments to other functions. The \texttt{twice} function's first argument is a function \texttt{f} which takes something of type \texttt{a} and produces something of type \texttt{a}. The second argument \texttt{x} is then something of type \texttt{a}. By not specifying a type and instead using a type variable, \texttt{twice} can be used on any type, as long as \texttt{x} has the same type as in the input to \texttt{f} and that \texttt{f} does not change the type of \texttt{x}.
    
    In addition to being first-class, functions in \Idris are \textit{pure} meaning that they do not have side effects (e.g. they do not do console I/O and they do not throw exceptions). This in turn means that for the same input, a function will always give the same output \cite{brady_2017}.


\section{Local variables}
    In function declarations we might not want to clutter the equation and instead have local variables which clearly describe what the various parts are. In \Idris you can use the keywords `\texttt{let}' and `\texttt{in}' to define such variables.
    
    \begin{code}[label={idr:let-eg}, caption={A program which returns the longer of two given words}]
        longer : String -> String -> String
        longer word1 word2 =
          let
            len1 = length word1
            len2 = length word2
          in
            if len1 > len2 then word1 else word2
    \end{code}
    
    The variables \texttt{len1} and \texttt{len2} exist only in the statement after the \texttt{in} keyword. If we were to write the program in Listing \ref{idr:let-eg} without a \texttt{let}/\texttt{in} definition it would look like this:
    
    \begin{code}[caption={Listing \ref{idr:let-eg} without using \texttt{let}/\texttt{in}}]
    longer : String -> String -> String
    longer word1 word2 =
      if (length word1) > (length word2) then word1 else word2
    \end{code}

    This is still mostly readable, but less so than Listing \ref{idr:let-eg}.
    

\section{Data Types}
    In addition to the given types, we can define our own data types. This is useful when we want to capture an aspect of the program we are writing which is likely to not be pre-defined in the \Idris prelude. Data types are defined using the \texttt{data} keyword. A data type is defined by a type constructor and one or more data constructors \cite{brady_2017}. Some of the included types are defined as data types, for example \texttt{List}.
    
    \begin{code}[caption={\texttt{List} as defined in the \Idris prelude}]
        data List : (elem : Type) -> Type where
            Nil  : List elem
            (::) : (x : elem) -> (xs : List elem) -> List elem 
    \end{code}
    A \texttt{List} of elements of type \texttt{Type} can be constructed by either constructing the empty list \texttt{Nil}, or an \texttt{x} of type \texttt{elem} (the type of all elements in the list) followed by a list of more elements of the same type.
    
    \begin{code}[caption={\texttt{Bool} as defined in the \Idris prelude}]
        data Bool = False
                  | True
    \end{code}
    Data types can be constructed with either equals and vertical bars (see the definition of \texttt{Bool} just above) or as functions, like in the definition of \texttt{List}. They are mostly interchangeable, but the \texttt{data ... -> Type where} syntax is more flexible and general, at the cost of also being more verbose.
    \\
    
    Data type definitions can also be recursive.


\section{Natural Numbers}\label{des:nats}
    \begin{code}[caption={Natural numbers as defined in the \Idris prelude}]
        data Nat = Z | S Nat
    \end{code}

    Natural numbers in \Idris are also a data record. They base case is the constant 0 (\texttt{Z} in Idris) and the other constructor is the \textit{successor function} \texttt{S}. From these two, all the natural numbers can be constructed:
    
    \begin{tabular}{r l}
        \texttt{Z} & $\mapsto 0$ \\ 
        \texttt{S Z} & $\mapsto 1$ \\ 
        \texttt{S (S Z)} & $\mapsto 2$ \\
        etc & \\
        \vdots & \\
    \end{tabular}
    \par
    The benefit of this is that it allows us to pattern match on numbers.
    
    \begin{code}[caption={Pattern matching on \texttt{Nat}s}]
        natFact : Nat -> Nat
        natFact Z = 1
        natFact (S k) = (S k) * fact k
    \end{code}
    By pattern matching on the possible constructors for \texttt{Nat}, we have captured the base case, and the recursive step for calculating the factorial of a number using the \texttt{natFact} function.
    Pattern matching also helps in terms of decidability.
    \\


\section{The \texttt{mutual} declaration}
    Occasionally, there might be a situation where we need to define functions or types which rely on each other. This is problematic because in \Idris, something must be defined before you can use it. The \texttt{mutual} declaration allows us to explicitly define things which depend on each other, thereby `bypassing' the declaration order.
    
    \begin{code}[caption={Mutually declaring \texttt{odd} and \texttt{even}}]
        mutual
            odd : Nat -> Bool
            odd Z     = False
            odd (S k) = even k
            
            even : Nat -> Bool
            odd Z     = True
            odd (S k) = odd k
    \end{code}
    
    If we wanted to define two functions \texttt{odd} and \texttt{even} which determined the parity of a natural number, we could define them by saying that zero is even, and a number bigger than that has the opposite parity of its predecessor. However, in doing so, \texttt{odd} relies on the existence of the \texttt{even} function and vice-versa, so both would have to be defined before each other. Since it is impossible to define two things at the same time, we use a \texttt{mutual} block to indicate that the functions rely on each other and that only when the entire \texttt{mutual} block has been defined will the functions work correctly.


\section{\texttt{Maybe} -- Handling uncertainty}\label{idris:maybe}
    \begin{code}[caption={The \texttt{Maybe} data type}]
        data Maybe : (a : Type) -> Type where
            Nothing : Maybe a
            Just : (x : a) -> Maybe a
    \end{code}

    The \texttt{Maybe} data type captures the concept of failure or uncertainty of a function. Either, the function will have \texttt{Just} a value \texttt{x} or \texttt{Nothing}. We could use this to implement taking the n$^{th}$ element from a list.
    \begin{code}[caption={List indexing using \texttt{Maybe}}]
        getIndex : Nat -> List a -> Maybe a
        getIndex _ Nil = Nothing
        getIndex Z (x :: xs) = Just x
        getIndex (S k) (x :: xs) = getIndex k xs
    \end{code}
    Since we could either be given the empty list or run out of bounds, we use \texttt{Maybe} to reflect that the index may not be valid. If the index exists, we return \texttt{Just x} where \texttt{x} is the n$^{th}$ element, and if we run out of bounds we return \texttt{Nothing}.
